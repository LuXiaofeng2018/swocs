\documentclass[a4paper,12pt]{report}

\usepackage[utf8]{inputenc}
\usepackage[english]{babel}
\usepackage[dvips]{graphicx,color}
\usepackage{calc}
\usepackage{multirow}
\usepackage{colortbl}
\usepackage{xcolor}
\usepackage{moreverb}
\usepackage{hyperref}
\hypersetup{colorlinks=false}
\usepackage[all]{hypcap}

\newcommand{\fig}[3]
{
	\begin{figure}[ht!]
		\centering
		\includegraphics[scale=0.4]{#1}
		\caption{#2.\label{#3}}
	\end{figure}
}

\newcommand{\PICTURE}[4]
{
	\begin{figure}[ht!]\centering\begin{picture}(#1)#2\end{picture}
	\caption{#3.\label{#4}}\end{figure}
}

\newcommand{\TABLE}[4]
{
	\begin{table}[ht!]\centering
	\begin{tabular}{#1}\hline#2\\\hline\end{tabular}
	\caption{#3.\label{#4}}\end{table}
}

\title
{
	{\bf\Large User manual for \emph{SWOCS}}\\
	{\large	version 1.2}
}
\author
{
	{\bf Authors:} Javier Burguete Tolosa and Samuel Ambroj Pérez\\
	{\small Copyright~\copyright~2011-2012 Javier Burguete Tolosa.
	 All right reserved.}
}
\date{\today}

\hyphenation
{
	furrow
}

\newcommand{\swocs}{\emph{SWOCS}}
\newcommand{\IT}[1]{{\sl ``#1''}}

\begin{document}

\maketitle

\tableofcontents

\setlength{\parskip}{\baselineskip / 2}

\chapter{Introduction}

The program {\swocs} (\emph{Shallow Water Open Channel Simulator}) is designed
to solve numerically one-dimensional transient flows with solute transport and
infiltration in a channel or furrow. There are two versions of the code for the
operating systems Unix and Windows. Windows is a trademark of Microsoft
Corporation.

The code has been written in C language and is freely distributed under a BSD
type license.

\section{Internationalization}

In order to avoid conflicts with formats in the input and output data files when
using different regional configurations, the real numbers format has been forced
to the international standard. Therefore, the character \IT{.} has to be used to
indicate the decimal point. The character \IT{e}, indicates the exponential of
the number. On the other hand, all units have to be specified in the
International Unit System.

\chapter{Installation and execution instructions}

\section{Compilation requirements}

In order to compile the program {\swocs} on Unix operating systems, it is
necessary to have installed the following packages:
\begin{verbatimtab}
	gcc, GNU make
\end{verbatimtab}
The compilation can be made under \emph{cygwin} (\url{http://www.cygwin.com/})
and \emph{MinGW} (\url{http://www.mingw.org/}) platforms for Windows users.

It is also possible to use different compilers. In this case, the included
compilation file ({\tt makefile}) does not work, and it would be necessary
to create some kind of file or project to compile the set of source code files.  
\section{Installation on Unix systems}

The following steps are required to install this program on Unix systems:
\begin{enumerate}
	\item Decompress the file \emph{swocs-src-1-0.zip}. \\
		Once decompressed, the contents of the folder must be the following one:
		some files with {\tt c} extension, some files with {\tt h} extension, a
		compilation file ({\tt makefile}) in \emph{GNU make} format.
	\item Execute the shell instruction: {\tt make} \\
			The executable {\tt swocs} is built. 
	\item Some manuals can be also downloaded:
		\begin{description}
			\item{Spanish manual} ({\it swocs-manual-de-usuario-1-0.pdf}),
			\item{English manual} ({\it swocs-user-manual-1-0.pdf})
			\item and a programmer's reference manual 
				({\it swocs-reference-manual-1-0.pdf}). 
		\end{description}
\end{enumerate}

\section{Installation on Microsoft Windows systems}

The following step is required to install this program on Windows systems:
\begin{enumerate}
	\item Download the executable \emph{swocs.exe}. \\
		If this file is downloaded, no more files will be necessary and no
		compilation will be required.
	\item It is also possible to compile the source code, being necessary to
		download the file \emph{swocs-src-1-0.zip} and proceed as detailed in
		the previous section.
	\item Some manuals can be also downloaded:
		\begin{description}
			\item{Spanish manual} ({\it swocs-manual-de-usuario-1-0.pdf}),
			\item{English manual} ({\it swocs-user-manual-1-0.pdf})
			\item and a programmer's reference manual 
				({\it swocs-reference-manual-1-0.pdf}). 
		\end{description}
\end{enumerate}

\section{Program execution}

In Microsoft Windows systems, to execute the program go to the commands interface:
\begin{description}
\item[Start$\rightarrow$Programs$\rightarrow$Accesories$\rightarrow$Command Prompt]
\end{description}
Then, access to the folder where the executable file is located using the \emph{cd} command.
\begin{description}
\item[Example:] cd C:$\backslash$Programs$\backslash$Swocs
\end{description}

The execution syntax, for Unix and Microsoft Windows systems, is the following one:
\begin{description}
\item \emph{swocs}  \emph{file1} \emph{file2} \emph{file3} \emph{file4} \emph{file5} \emph{file6}
\end{description}
where the files are:
\begin{description}
\item[\it file1,] input channel file.
\item[\it file2,] output variables file.
\item[\it file3,] output flows file.
\item[\it file4,] output advance file.
\item[\it file5,] input probes file.
\item[\it file6,] output probes file.
\end{description}

File formats are explained in the following chapter.

\chapter{Format of input and output files}

ASCII character encoding is used for input and output files. These files can be created and modified
with a wide range of text editors ({\tt notepad}, {\tt vim}). 


\section{Format of input channel file}
An example of the input channel file is shown:

{\footnotesize
\begin{boxedverbatim}
100 0 0.14 1.222 0.27
2
1 1 1
0.062
9.6e-06 0.115 0 0.8
11.82
1
0 0.001
4
567 0
567 0.0106
867 0.0106
867 0
401 1
3900 60 0.9 0.01
2 2
1
\end{boxedverbatim}
}

Each value has the following correspondence:

{\footnotesize
\begin{boxedverbatim}
(channel length) (channel slope) (bottom width) (wall slope) (channel depth)
(outlet type)
(friction model) (infiltration model) (diffusion model)
(friction coefficients) ...
(infiltration coefficients) ...
(diffusion coefficients) ...
(number of points of the inlet water hydrogam)
(time of the inlet water hydrogam) (discharge of the inlet water hydrogram)
...
(number of points of the inlet solute hydrogam)
(time of the inlet solute hydrogam) (discharge of the inlet solute hydrogram)
...
(number of mesh nodes) (initial conditions type)
(final time) (time interval between measures) (cfl number) (minimum depth)
(numerical model of surface flow) (numerical model of diffusion)
(physical model of surface flow)
\end{boxedverbatim}
}

\subsection{Outlet type}
\begin{verbatimtab}
	1: Closed.
	2: Open.
\end{verbatimtab}

\subsection{Friction model and coefficients}
It is only available the Gauckler-Manning friction model in the current version. The only necessary coefficient for this model is the Gauckler-Manning number, $n$.
More than a single number could be necessary for different friction models. 
\begin{verbatimtab}[4]
(friction model): 
		1: Gauckler-Manning.
(friction coefficients):
\end{verbatimtab}
\vspace{-0.60cm}
\hspace{1.8cm}$n$

The formulation used for the Gauckler-Manning model (\cite{JaviSurcos1}):
\begin{equation}
S_f=\frac{n^2 Q|Q|P^{4/3}}{A^{10/3}}
\end{equation}
\noindent where $S_f$ is the friction slope, $Q$ is the superficial discharge, $P$ is the superficial cross sectional wetted perimeter
and $A$ is the superficial cross sectional wetted area.

\subsection{Infiltration model and coefficients}
It is only available the Kostiakov-Lewis infiltration model in the current version. Four infiltration coefficients need to be defined for this model.
A different number of coefficientes could be necessary for different infiltration models.  
\begin{verbatimtab}[4]
(infiltration model): 
		1: Kostiakov-Lewis.
(infiltration coefficients):
\end{verbatimtab}
\vspace{-0.5cm}
\hspace{1.8cm}$K$\hspace{0.9cm}$a$\hspace{0.9cm}$i_c$\hspace{0.9cm}$D$

The formulation used for the Kostiakov-Lewis model (\cite{JaviSurcos1}):
\begin{equation}
\frac{d\alpha}{dt}=P\left[ i_c + Ka \left( \frac{\alpha}{DK} \right)^{a-1/a} \right]
\end{equation}
\noindent where $\alpha$ is the infiltrated cross sectional area, $t$ is the temporal coordinate, $i_c$ is the saturated infiltration long-term rate, $K$ is the Kostiakov infiltration parameter, $a$ is the Kostiakov infiltration exponent and $D$ is the distance between furrows.

\subsection{Diffusion model and coefficients}
It is only available the Rutherford diffusion model in the current version. The only necessary coefficient for this model is 
the adimensional superficial diffusion coefficient, $D_x$.
\begin{verbatimtab}[4]
(diffusion model): 
		1: Rutherford.
(diffusion coefficients):
\end{verbatimtab}
\vspace{-0.5cm}
\hspace{1.8cm}$D_x$ 

The Rutherford model formulation for the infiltration (\cite{JaviSurcos1}):
\begin{equation}
K_x=D_x\sqrt{gAP|S_f|}
\end{equation}
\noindent where $K_x$ is the superficial diffusion coefficient and $g$ is the gravitational constant.

\subsection{Initial conditions type}
\begin{verbatimtab}[4]
		1: Dry bed.
\end{verbatimtab}

\subsection{Numerical model of surface flow}
\begin{verbatimtab}
	1: McCormack.
	2: Upwind.
\end{verbatimtab}

\subsection{Numerical model of diffusion}
\begin{verbatimtab}
	1: Explicit.
	2: Implicit.
\end{verbatimtab}

\subsection{Physical model of surface flow}
\begin{verbatimtab}
	1: Complete.
	2: Zero-inertia.
	3: Diffusive.
	4: Kinematic.
\end{verbatimtab}

\section{Format of output variables file}
Five different variables are saved for every longitudinal coordinate, $x$, in the final simulation time. 
The order of every row:

\hspace{1.8cm}$x$\hspace{0.9cm}$A$\hspace{0.9cm}$Q$\hspace{0.9cm}$A\cdot s$\hspace{0.9cm}$\alpha$\hspace{0.9cm}$\alpha\cdot s_i$ 

\hspace{3.8cm} $\cdots \cdots \cdots$

\noindent where $s$ is the cross-sectional average solute concentration and $s_i$ is the infiltrated solute concentration.

\section{Format of output flows file}
Four different variables are saved for every longitudinal coordinate, $x$, in the final simulation time. 
The order of every row:

\hspace{1.8cm}$x$\hspace{0.9cm}$\frac{\delta(Qu)}{\delta x}$\hspace{0.9cm}$-gA\frac{\delta z_b}{\delta x}$\hspace{0.9cm}$gA\frac{\delta h}{\delta x}$
\hspace{0.9cm}$gAS_f$ 

\hspace{3.8cm} $\cdots \cdots \cdots$

\noindent where $u$ is the cross-sectional averaged velocity, $h$ is the superficial water depth and $z_b$ is the 
level of the lowest point in the cross section.

\section{Format of output advance file}

The pair of values ($t$, $x_{av}$) are saved in every time step:

\hspace{1.8cm}$t$\hspace{0.9cm}$x_{av}$

\hspace{1.8cm} $\cdots$

\noindent where $x_{av}$ is the maximum longitudinal coordinate reached by water in time $t$.

\section{Format of input probes file}
 
The total number of probes and the longitudinal location $x$ of every probe need to be defined:

\begin{verbatimtab}
	(total number of probes)
	(longitudinal coordinate)
	...
\end{verbatimtab}

\section{Format of output probes file}
The number of columns depends on the number of probes defined in the input probes file.
The pair of values ($h$, $s$) for every probe are saved in every time step:

\hspace{1.8cm}$t$\hspace{0.9cm}$h_1$\hspace{0.9cm}$s_1$\hspace{0.9cm}$\cdots$\hspace{0.9cm}$h_N$\hspace{0.9cm}$s_N$

\hspace{3.8cm} $\cdots \cdots$

\noindent where $N$ is the number of probes. 

\clearpage
\section*{Notation}

\begin{description}
\item $\alpha$ = infiltrated cross sectional area,
\item $A$ = superficial cross sectional wetted area,
\item $a$ = Kostiakov infiltration exponent,
\item $D$ = distance between furrows,
\item $D_x$ = adimensional superficial diffusion coefficient,
\item $g$ = gravitational constant,
\item $h$ = superficial water depth, 
\item $i_c$ = saturated infiltration long-term rate, 
\item $K$ = Kostiakov infiltration parameter, 
\item $K_x$ = superficial diffusion coefficient,
\item $N$ = number of probes, 
\item $n$ = Gauckler-Manning roughness coefficient,
\item $P$ = superficial cross sectional wetted perimeter,
\item $Q$ = superficial discharge,
\item $S_f$ = friction slope,
\item $s$ = cross-sectional average solute concentration,
\item $s_i$ = infiltrated solute concentration,
\item $t$ = temporal coordinate,
\item $u=\frac{Q}{A}$ = cross-sectional averaged velocity,
\item $x$ = longitudinal coordinate,
\item $x_{av}$ = maximum longitudinal coordinate reached by water in time $t$,
\item $z_b$ = level of the lowest point in the cross section,
\end{description}


\clearpage
\begin{thebibliography}{}

\bibitem[Burguete et~al., 2009]{JaviSurcos1}
Burguete, J., Zapata, N., Garc\'{\i}a-Navarro, P., Ma\protect{\"{\i}}kaka, M.,
  Play\'an, E., and Murillo, J. (2009).
\newblock Fertigation in furrows and level furrow systems. \protect{I}: Model
  description and numerical tests.
\newblock {\em ASCE Journal of Irrigation and Drainage Engineering},
  135(4):401--412.

\end{thebibliography}

\end{document}

